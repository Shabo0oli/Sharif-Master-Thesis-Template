\chapter{ویژگی هار} 
\label{app:haar}
در این بخش ویژگی هار به اختصار معرفی خواهد شد. این ویژگی در ابتدا برای بازیابی تصویر در   \cite{Tieu2004} استفاده شد و بعد از آن به دفعات در کاربردهای دیگر بینایی ماشین از آن استفاده شده است. 

برای  استخراج ویژگی هار مجموعه‌ای  از چندین فیلتر خطی مرتبه اول مانند شکل \ref{fig:linear-filters} در نظر می‌گیریم.
\begin{figure}[ht]
\centering
\includegraphics[width=7cm]{linear-filters}
\caption {\small {چند نمونه از فیلترهای خطی استفاده شده برای استخراج ویژگی}}
\label{fig:linear-filters}
\end{figure}
برای استخراج ویژگی از روی یک عکس به صورتی که در شکل \ref{fig:haar} نشان داده شده است تمامی این ۲۵ فیلتر با عکس مورد نظر کانولوشن و سپس \inpdic{نمونه‌برداری کاهشی}{Down-Sample} می‌شود. این کار تا سه مرحله انجام می‌شود و در نهایت میانگین رنگ پیکسل هرکدام از ۱۵۶۲۵ عکس به دست آمده در یکی از درایه‌های بردار ویژگی نهایی قرار می‌گیرد. 
\begin{figure}[ht]
\centering
\includegraphics[width=13cm]{haar}
\caption {\small {نمایش شماتیک نحوه استخراج ویژگی هار از روی عکس}}
\label{fig:haar}
\end{figure}
در این روش هدف از ترکیب فیلتر‌های ساده ساختن فیلترهایی است که ویژگی‌های خاصی از تصاویر را استخراج کنند مثلاً ترکیب سه فیلتر زیر، شکل \ref{fig:haar2} به نوارهای رنگی در شکل حساس بوده و مقدار بیشتری به عکس ببر اختصاص می‌دهد.
\begin{figure}
\centering
\includegraphics[width=5cm]{haar2}
\caption{\small {نتیجه اعمال فیلترهای مختلف بر روی دو عکس مختلف. همان طور که دیده می‌شود ترکیب این سه فیلتر به نوارهای رنگی حساس بوده و مقدار بیشتری به عکس ببر اختصاص می‌دهد.}}
\label{fig:haar2}
\end{figure}
